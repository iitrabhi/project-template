\documentclass[preprint,review,10pt]{elsarticle}
% Packages ------------------------------------------------------------------
    \usepackage[a4paper,left=2.5cm,right=2.5cm,top=3cm,bottom=3cm]{geometry}
    \usepackage[right]{showlabels} % Comment this before submitting
    \usepackage{lineno}
    %--------------------------------------------------------------------------
    % Essential Packages
    %--------------------------------------------------------------------------
    \usepackage[parfill]{parskip}
    \usepackage[english]{babel}
    \usepackage{lmodern, amstext, amssymb, bm, amsthm, mathtools, xcolor, graphicx}
    \usepackage[format=plain, labelfont={bf,it}, textfont=it]{caption}
    \usepackage{subcaption, enumitem, booktabs, tabu, tabularx, algpseudocode, algorithm2e}
    %--------------------------------------------------------------------------
    %% Custom Commands
    %--------------------------------------------------------------------------
    \newcommand{\disp}{\boldsymbol{u}}
    \newcommand{\xvec}{\boldsymbol{x}}
    \newcommand{\dd}[1]{\mathrm{d}#1}
    %--------------------------------------------------------------------------
    %% Bibliography Styles
    %--------------------------------------------------------------------------
    \usepackage{float}
    \usepackage{natbib}
    \bibliographystyle{unsrtnat}
    \usepackage[bookmarks=true,colorlinks=true,linkcolor=blue,citecolor=red]{hyperref}
    \pdfstringdefDisableCommands{\let\uppercase\relax}
    \makeatletter
    \AtBeginDocument{\def\@citecolor{red}}
    \makeatother
    \biboptions{sort&compress}
    %--------------------------------------------------------------------------
    \usepackage[nameinlink,capitalize]{cleveref}
    \journal{Elsevier}
    
\begin{document}

% Front Matter --------------------------------------------------------------

\begin{frontmatter}

\title{The title of the research article}
    
    \author[1]{Author One}
    \author[1]{Author Two}
    \author[2]{Author Three\texorpdfstring{\corref{cor1}}{}}
    \cortext[cor1]{Corresponding author. Email: email@email.com}
    

    \affiliation[1]{%
     Organization One, Organization Address
    }
    
    \affiliation[2]{%
      Organization Two, Organization Address
    }
        

\begin{highlights}
\item Please list your hypotheses for the paper in bullet points, clearly stating the expected relationships and outcomes you predict to observe in your numerical experiments. Without a clear hypothesis, the study could lack direction and purpose, making it challenging to draw meaningful conclusions or contribute effectively to the body of scientific knowledge.
\item ... 

\end{highlights}

\begin{abstract}
% Abstract content here
In draft phase, it’s essential to concisely summarize your research using the following structure:

\textbf{Motivation:} Begin by stating the background of your study. Explain what prompted you to explore this topic and why it’s important within your field.

\textbf{Purpose:} Clearly define the main objective of your paper. What specific problem are you addressing, and what do you aim to achieve?

\textbf{Methodology:} Briefly describe the methods or approaches you used to conduct your research. Include any significant techniques, tools, or processes that were crucial to your study.

\textbf{Originality:} Highlight what makes your methodology or approach unique. Explain how it differs from existing studies and contributes new insights or perspectives.

\textbf{Value:} Articulate the significance of your work. Discuss the potential impact, applications, or benefits that arise from your original contributions.

\textbf{Findings:} Summarize the key results or conclusions of your research. Provide a snapshot of what you discovered and its relevance to the field.


\end{abstract}



\begin{keyword}
% Keywords here
Keyword1 \sep Keyword2 \sep Keyword3
\end{keyword}

\end{frontmatter}

\linenumbers
\section{Introduction}
In recent studies, significant progress has been made in the field of computational mechanics. For instance, the application of generic methodologies has been extensively discussed in the literature. The work by \citet{LoremIpsum2015a} presents a foundational approach to computational software development, which has paved the way for more specialized studies. Following this, \citet{LoremIpsum2023a} introduced efficient solutions for handling complex placeholder problems on adaptive grids. Moreover, the research by \citet{LoremIpsum2023b} explores innovative machine learning frameworks for addressing fictitious scenarios, demonstrating the adaptability of modern computational techniques to a broad range of applications.

\section{Formulation}
\section{Numerical experiments}
\section{Conclusion}

% \section{Introduction}
% \section{Formulation}
% \section{Numerical experiments}
% \section{Conclusion}

\bibliography{bibliography}
\end{document}
